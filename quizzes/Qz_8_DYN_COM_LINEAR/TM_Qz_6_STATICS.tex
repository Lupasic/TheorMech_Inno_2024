\documentclass[aspectratio=169,xcolor=table,10pt, notes=hide]{beamer}


\usetheme[faculty=phil]{fibeamer}
\usepackage{polyglossia}

\setmainlanguage{russian} %% main locale instead of `english`, you
%% can typeset the presentation in either Czech or Slovak,
%% respectively.
\setotherlanguages{english} %% The additional keys allow
%%
%%   \begin{otherlanguage}{czech}   ... \end{otherlanguage}
%%   \begin{otherlanguage}{slovak}  ... \end{otherlanguage}
%%
%% These macros specify information about the presentation
\title[Theoretical Mechanics]{Theoretical Mechanics, Quiz 6: STATICS} %% that will be typeset on the
\subtitle{Statics \\ \  \\ \  } %% title page.
\author{Oleg Bulichev}
%% These additional packages are used within the document:
\usepackage{ragged2e}  % `\justifying` text
\usepackage{booktabs}  % Tables
\usepackage{tabularx}
\usepackage{tikz}      % Diagrams
\usetikzlibrary{decorations.pathreplacing,calligraphy,calc,graphs, shapes, backgrounds}
\usepackage{amsmath, amssymb}
\usepackage{url}       % `\url`s
\usepackage{listings}  % Code listings
\usepackage{floatrow}
\usepackage{mathtools}
\usepackage{fontspec}
\usepackage{multicol}
\usepackage{pdfpages}
\usepackage{wrapfig}
\usepackage{animate}
\usepackage{booktabs}
\usepackage{multirow}
\usepackage{multimedia}
\usepackage{makecell}
\usepackage{colortbl}
\usepackage{hhline}
\usepackage{rotating}
\usepackage{amsmath}

\usepackage[font={}, labelfont=it,textfont={it},justification=centering, skip=0pt]{caption}
% will apply to all subcaptions
\usepackage[font={},skip=2pt]{subcaption}


\graphicspath{{resources/}}
\frenchspacing




% \setbeamertemplate{caption}[numbered]
\captionsetup[figure]{labelformat=empty}


\newcommand{\fbckg}[1]{\usebackgroundtemplate{\includegraphics[width=\paperwidth]{#1}}}%frame background

\usepackage[framemethod=TikZ]{mdframed}
\newcommand{\dbox}[1]{
\begin{mdframed}[roundcorner=3pt, backgroundcolor=yellow, linewidth=0]
\vspace{1mm}
{#1}
\vspace{1mm}
\end{mdframed}
}
\addtobeamertemplate{frametitle}{}{\vspace{-0.35cm}}

% \usepackage{pgfpages}
% \pgfpagesuselayout{4 on 1}[a4paper,border shrink=2mm,landscape]
\usepackage{color}
\usepackage{rotating}
\usepackage{tabularray}

\begin{document}
\setlength{\abovedisplayskip}{0pt}
\setlength{\belowdisplayskip}{0pt}
\setlength{\abovedisplayshortskip}{0pt}
\setlength{\belowdisplayshortskip}{0pt}

\fbckg{fibeamer/figs/title_page.png}
\frame[c]{\setcounter{framenumber}{0}
    \usebeamerfont{title}%
    \usebeamercolor[fg]{title}%
    \begin{minipage}[b][6.3\baselineskip][b]{\textwidth}%
        \textcolor{black}{\raggedright\inserttitle}
    \end{minipage}
    % \vskip-1.5\baselineskip

    \usebeamerfont{subtitle}%
    \usebeamercolor[fg]{framesubtitle}%
    \begin{minipage}[b][3\baselineskip][b]{\textwidth}
        \raggedright%
        \insertsubtitle%
    \end{minipage}
    \vskip.25\baselineskip
}

\note{}

\fbckg{fibeamer/figs/common.png}

\begin{frame}[t]{Quiz 6}
  \begin{minipage}{0.6\textwidth}
    The circle with uniformly distributed length weight $P= 40$ and radius $R=1$ lies on the brickwork. Distance between the walls $l=1.6$. 
    \medskip

    Disregarding friction, find the circle pressure on the brickwork in points $A$ and $B$.
  \end{minipage}
  \begin{minipage}{0.39\textwidth}
    \begin{figure}[H]
      \centering\includegraphics[height=6cm,width=1\textwidth,keepaspectratio]{quiz6_1}
      \caption*{Quiz 6, Task 1}
    \end{figure}
  \end{minipage}
\end{frame}

\begin{frame}[t]{Quiz 6: solution}
\framesubtitle{}
  \begin{columns}[T,onlytextwidth]
    \begin{column}{0.49\textwidth}
      \begin{enumerate}
        \item Find $\alpha$. $$\cos \alpha = \frac{l}{2R},\ \sin \alpha = \sqrt{1- \left(\frac{l}{2R}\right)^2}$$
        \item Equilibrium equations: $$ \left\{\begin{matrix*}[l]
        N_a \cos \alpha - N_b \cos \alpha = 0 \\
        N_a \sin \alpha + N_b \sin \alpha - P = 0
        \end{matrix*}\right. $$
        \item Solution: $$ \left\{\begin{matrix*}[l]
        N_a = N_b 
        \\ 2N_a \sin \alpha = P
        \end{matrix*}\right.$$
        $$ N_a=N_b = \frac{P}{2\sqrt{1-\left(\frac{l}{2R}\right)^2}}=33.3 $$
      \end{enumerate}
    \end{column}
    \begin{column}{0.49\textwidth}
      \begin{figure}[H]
        \centering\includegraphics[height=6cm,width=1\textwidth,keepaspectratio]{quiz6_1_sol.png}
        % \caption{caption_name}
        \label{fig:quiz6_1_sol.png}
      \end{figure}
    \end{column}
  \end{columns}
\end{frame}

\end{document}

