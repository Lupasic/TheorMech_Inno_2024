\documentclass[aspectratio=169,xcolor=table,10pt, notes=hide]{beamer}


\usetheme[faculty=phil]{fibeamer}
\usepackage{polyglossia}

\setmainlanguage{russian} %% main locale instead of `english`, you
%% can typeset the presentation in either Czech or Slovak,
%% respectively.
\setotherlanguages{english} %% The additional keys allow
%%
%%   \begin{otherlanguage}{czech}   ... \end{otherlanguage}
%%   \begin{otherlanguage}{slovak}  ... \end{otherlanguage}
%%
%% These macros specify information about the presentation
\title[Theoretical Mechanics]{Theoretical Mechanics, Quiz 1: PART LIN} %% that will be typeset on the
\subtitle{Particle kinematics \\ Linear Algebra \\ \  } %% title page.
\author{Oleg Bulichev}
%% These additional packages are used within the document:
\usepackage{ragged2e}  % `\justifying` text
\usepackage{booktabs}  % Tables
\usepackage{tabularx}
\usepackage{tikz}      % Diagrams
\usetikzlibrary{decorations.pathreplacing,calligraphy,calc,graphs, shapes, backgrounds}
\usepackage{amsmath, amssymb}
\usepackage{url}       % `\url`s
\usepackage{listings}  % Code listings
\usepackage{floatrow}
\usepackage{mathtools}
\usepackage{fontspec}
\usepackage{multicol}
\usepackage{pdfpages}
\usepackage{wrapfig}
\usepackage{animate}
\usepackage{booktabs}
\usepackage{multirow}
\usepackage{multimedia}
\usepackage{makecell}
\usepackage{colortbl}
\usepackage{hhline}
\usepackage{rotating}
\usepackage{amsmath}

\usepackage[font={}, labelfont=it,textfont={it},justification=centering, skip=0pt]{caption}
% will apply to all subcaptions
\usepackage[font={},skip=2pt]{subcaption}


\graphicspath{{resources/}}
\frenchspacing




% \setbeamertemplate{caption}[numbered]
\captionsetup[figure]{labelformat=empty}


\newcommand{\fbckg}[1]{\usebackgroundtemplate{\includegraphics[width=\paperwidth]{#1}}}%frame background

\usepackage[framemethod=TikZ]{mdframed}
\newcommand{\dbox}[1]{
\begin{mdframed}[roundcorner=3pt, backgroundcolor=yellow, linewidth=0]
\vspace{1mm}
{#1}
\vspace{1mm}
\end{mdframed}
}
\addtobeamertemplate{frametitle}{}{\vspace{-0.35cm}}

% \usepackage{pgfpages}
% \pgfpagesuselayout{4 on 1}[a4paper,border shrink=2mm,landscape]
\usepackage{color}
\usepackage{rotating}
\usepackage{tabularray}

\begin{document}
\setlength{\abovedisplayskip}{0pt}
\setlength{\belowdisplayskip}{0pt}
\setlength{\abovedisplayshortskip}{0pt}
\setlength{\belowdisplayshortskip}{0pt}

\fbckg{fibeamer/figs/title_page.png}
\frame[c]{\setcounter{framenumber}{0}
    \usebeamerfont{title}%
    \usebeamercolor[fg]{title}%
    \begin{minipage}[b][6.3\baselineskip][b]{\textwidth}%
        \textcolor{black}{\raggedright\inserttitle}
    \end{minipage}
    % \vskip-1.5\baselineskip

    \usebeamerfont{subtitle}%
    \usebeamercolor[fg]{framesubtitle}%
    \begin{minipage}[b][3\baselineskip][b]{\textwidth}
        \raggedright%
        \insertsubtitle%
    \end{minipage}
    \vskip.25\baselineskip
}

\note{}

\fbckg{fibeamer/figs/common.png}

\begin{frame}[t]{Quiz 1}
    \begin{minipage}{0.6\textwidth}
        \begin{enumerate}
            \item Write down a velocity equation, using natural form. Show up the dimensions of each element of the formulae. Explain what each element means and their properties, if exists.
            \item Find a transformation matrix from $X0Y$ to $X'0'Y'$ (length of vectors is important).
        \end{enumerate}
    \end{minipage}
    \begin{minipage}{0.35\textwidth}
        \vspace{-0.5cm}
        \begin{figure}[H]
            \centering
            \begin{tikzpicture}
                % Include the image in a node
                \node [above right, inner sep=0] (image) at (0,0)
                {\centering\includegraphics[height=5cm,width=1\textwidth,keepaspectratio]{white.png}};
                % Create scope with normalized axes
                \begin{scope}[
                        x={($ 0.1*(image.south east)$)},
                        y={($ 0.1*(image.north west)$)}]
                    % Grid and axes' labels
                    \draw[lightgray,step=1] (image.south west) grid (image.north east);
                    \foreach \x in {0,1,...,10} { \node [below] at (\x,0) {\x}; }
                    \foreach \y in {0,1,...,10} { \node [left] at (0,\y) {\y};}

                    % Labels
                    \draw[latex-, very thick,black] (6,1) -- (6,3)
                    node[above right,black]{$0$};
                    \node[below, black] at (6,1) {$X$};

                    \draw[latex-, very thick,black] (4,3) -- (6,3);
                    \node[below, black] at (4,3) {$Y$};

                    \draw[latex-, very thick,black] (5,7) -- (3,9)
                    node[above right,black]{$0'$};
                    \node[below, black] at (5,7) {$X'$};

                    \draw[latex-, very thick,black] (1,7) -- (3,9);
                    \node[below, black] at (1,7) {$Y'$};

                    \node[circle,fill=black,scale=0.5] at (6,3){};
                    \node[circle,fill=black,scale=0.5] at (3,9){};
                \end{scope}
            \end{tikzpicture}
            \caption*{Quiz 1, Task 2}
            \label{fig:file_name}
        \end{figure}
    \end{minipage}
\end{frame}

\begin{frame}[c]{Answer}
\framesubtitle{Task 2}
    \vspace*{-0.2cm}
    \begin{minipage}{0.6\textwidth}
        $T = \left[\begin{matrix}1 & 0 & -3\\0 & 1 & 1.5\\0 & 0 & 1\end{matrix}\right]$\\
        $R_z = \left[\begin{matrix}\cos{\left(45 \right)} & \sin{\left(45 \right)} & 0\\- \sin{\left(45 \right)} & \cos{\left(45 \right)} & 0\\0 & 0 & 1\end{matrix}\right] =\left[\begin{matrix}\frac{\sqrt{2}}{2} & \frac{\sqrt{2}}{2} & 0\\- \frac{\sqrt{2}}{2} & \frac{\sqrt{2}}{2} & 0\\0 & 0 & 1\end{matrix}\right]$ \\
        $Sc = \left[\begin{matrix}\sqrt{2} & 0 & 0\\0 & \sqrt{2} & 0\\0 & 0 & 1\end{matrix}\right]$ \\
        $H = T R_z Sc = \left[\begin{matrix}1 & 1 & -3\\-1 & 1 & 1.5\\0 & 0 & 1\end{matrix}\right]$
        Check --- $H \begin{bmatrix} 1\\ 0\\ 1 \end{bmatrix} = \begin{bmatrix} -2\\ 0.5\\ 1 \end{bmatrix}$
    \end{minipage}
    \begin{minipage}{0.35\textwidth}
        \vspace{-0.5cm}
        \begin{figure}[H]
            \centering
            \begin{tikzpicture}
                % Include the image in a node
                \node [above right, inner sep=0] (image) at (0,0)
                {\centering\includegraphics[height=5cm,width=1\textwidth,keepaspectratio]{white.png}};
                % Create scope with normalized axes
                \begin{scope}[
                        x={($ 0.1*(image.south east)$)},
                        y={($ 0.1*(image.north west)$)}]
                    % Grid and axes' labels
                    \draw[lightgray,step=1] (image.south west) grid (image.north east);
                    \foreach \x in {0,1,...,10} { \node [below] at (\x,0) {\x}; }
                    \foreach \y in {0,1,...,10} { \node [left] at (0,\y) {\y};}

                    % Labels
                    \draw[latex-, very thick,black] (6,1) -- (6,3)
                    node[above right,black]{$0$};
                    \node[below, black] at (6,1) {$X$};

                    \draw[latex-, very thick,black] (4,3) -- (6,3);
                    \node[below, black] at (4,3) {$Y$};

                    \draw[latex-, very thick,black] (5,7) -- (3,9)
                    node[above right,black]{$0'$};
                    \node[below, black] at (5,7) {$X'$};

                    \draw[latex-, very thick,black] (1,7) -- (3,9);
                    \node[below, black] at (1,7) {$Y'$};

                    \node[circle,fill=black,scale=0.5] at (6,3){};
                    \node[circle,fill=black,scale=0.5] at (3,9){};
                \end{scope}
            \end{tikzpicture}
            \caption*{Quiz 1, Task 2}
            \label{fig:file_name}
        \end{figure}
    \end{minipage}
\end{frame}

\end{document}

