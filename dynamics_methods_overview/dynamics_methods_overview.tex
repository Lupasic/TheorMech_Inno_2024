% \documentclass[a4paper]{article}
\documentclass[ 
   paper = A3, 
   paper = landscape, 
   pagesize = auto 
   ]{scrbook}

\usepackage{fullpage} % Package to use full page
\usepackage{parskip} % Package to tweak paragraph skipping
\usepackage{tikz} % Package for drawing
\usepackage{multicol}
\usepackage{hyperref}
\usepackage{graphicx} % for pdf, bitmapped graphics files
\graphicspath{{images/}} %path to images
\usepackage{amsmath} % assumes amsmath package installed
\usepackage{amssymb}  % assumes amsmath package installed
\usepackage{floatrow}
\usepackage{subcaption}
\usepackage{url}
\usepackage{lscape}
\usepackage{makecell}
\usepackage{booktabs}
\usepackage{color,colortbl}
\definecolor{Gray}{gray}{0.9}
\definecolor{LightGray}{gray}{0.95}
\usepackage{longtable}
\usepackage{array}
\title{Dynamics methods overview \\ Theoretical mechanics }
\author{Oleg Bulichev}

\begin{document}
% \begin{landscape}
\maketitle

\begin{longtable}{>{\centering\arraybackslash} m{6cm}
|>{\centering\arraybackslash} m{2cm}
|>{\centering\arraybackslash} m{3.5cm}
|>{\centering\arraybackslash} m{7.5cm}
|>{\centering\arraybackslash} m{6 cm}
|>{\centering\arraybackslash} m{9.5 cm}}

\caption{Methods overview}\\
\toprule
\toprule
\rowcolor{Gray}
\textbf{Method} & 
%<*top>
\textbf{Research object} & \textbf{Amount of equations (for one research object)} & \textbf{Equation} & \textbf{The best applications} & \textbf{Additional comments} \\
%</top>
\hline
\endfirsthead
\multicolumn{6}{c}%
{\tablename\ \thetable\ -- \textit{Continued from previous page}} \\
\hline
\rowcolor{Gray}
\textbf{Method} & \textbf{Research object} & \textbf{Amount of equations (for one research object)} & \textbf{Equation} & \textbf{The best applications} & \textbf{Additional comments}\\
\hline
\endhead
\hline \multicolumn{6}{r}{\textit{Continued on next page}} \\
\endfoot
\bottomrule
\bottomrule
\endlastfoot

2-nd Newtons law for inertial systems & 
%<*sndnewinert> 
Particle & 1--3 & $m\vec{a}=\sum\vec{F}$, where \par $m$ --- mass, $\vec{a}$ --- acceleration, $\vec{F}$ --- forces & For everything, if you can represent your system or body as a particle & \\
%</sndnewinert>
\rowcolor{LightGray}
2-nd Newtons law for non-inertial systems & 
%<*sndnewnoninert> 
Particle & 1--3 & $m\vec{a}_r=\sum\vec{F}+\vec{\Phi}_{cor}+\vec{\Phi}_{trans}$, \par where $\Phi$ is inertia force   & For everything, if you can represent your system or body as a particle & \begin{itemize}
    \item Inertial force is not real force, it is needed for compensating non-inertial nature of a system.
    \item Not obligatory that the particle should be on a body, like turning bus and a man. It can be Earth and satellite on Jupiter.
\end{itemize}  \\
%</sndnewnoninert>
Theorem on: \begin{enumerate}
    \item Motion of the centre of mass of a system
    \item Change of linear momentum of a system
\end{enumerate}  & 
%<*sndcomlinear> 
System & 1--3 & 
\begin{enumerate}
    \item $m\vec{a}_c=\sum\vec{F}; \ \vec{x}_c=\dfrac{\sum m_i\vec{x}_i}{\sum m_i}$
    \item $\dfrac{\mathrm{d} \vec{Q}_c}{\mathrm{d} t}=\sum\vec{F}; \ Q_c =\sum m\vec{v}_i$
\end{enumerate}
 & We are interested in linear motion. \begin{enumerate}
    \item Easy to find a displacement for a body of a system, motion equation for system, external forces.
    \item Easy to find a velocities for bodies.
\end{enumerate}  & \\
%</sndcomlinear> 
\rowcolor{LightGray}
Theorem on change of angular momentum of a system &
%<*sndangular> 
 System & 1 & \begin{eqnarray*}
    \frac{\mathrm{d} \vec{L}_c}{\mathrm{d} t}=\sum\vec{M}_c \\ \text{ $c$ -- point of calculation}\\
   \vec{L}_c =\sum \vec{L}_i,\ \vec{L}_i = J\vec{\omega} = \vec{Q}\times R
\end{eqnarray*} & We are interested in angular motion. Easy to find a angular velocities for bodies. & The choice of a point depends of the motion. If rotation -- more convenient to put it in the center of rotation, if planar -- in the center of mass.\\
%</sndangular>
d'Alembert principle (kinetostatics) & Body, system & 1--6 & \begin{eqnarray*}\left\{\begin{matrix}
\sum \vec{F} + \vec{\Phi} = 0\\
\sum \vec{M} + \vec{M}_{\Phi} = 0
\end{matrix}\right.
\end{eqnarray*} & To find reaction forces, if you know the motion equations of the system. & \begin{itemize}
    \item You can use it, when it is applicable to imagine in each time that the system is static.
    \item In contrast of the Coriolis or Transport forces of inertia, the d'Alambert force of inertia has not a special physical meaning. It's a math trick.
\end{itemize} \\
\rowcolor{LightGray}
Theorem on the change of kinetic energy of a system &
%<*sndkinen> 
Body, System & 1 & \begin{eqnarray*}
    \mathrm{d}T = \delta A\text{, or } T_2 - T_1 = A_{12} \\
    T_{lin} = \frac{mv^2}{2};\ T_{rot}=\frac{J\omega^2}{2}\\
    A_{12}=\int_{1}^{2}\delta A;\ A_{12}= \sum A_i\\ 
    \delta A=\vec{F} \cdot \delta\vec{r}= \vec{F}\delta\vec{r}cos(\vec{F} \hat{\ } \delta\vec{r})=|M|\delta\vec{\phi}\\
    \delta A = \Pi_1 - \Pi_2 = F\Delta h \text{ potential force}
\end{eqnarray*} & To find a correlation between displacement and velocity. Helpful if you need to find \textit{one} force.  & \begin{itemize}
    \item Work of internal forces may be \textbf{not equal to zero!} 
    \item $\delta\vec{r}$ can be changed on $\mathrm{d}\vec{r}$ because it is usually independent of time (scleronomic). This is done in order to be able to integrate by coordinates rather than by time.
\end{itemize}
 \\
%</sndkinen>
Principle of virtual:
\begin{enumerate}
    \item Displacements (work)
    \item Velocities
\end{enumerate}
 &
%<*sndpvd>  
 System & 1 &
 \begin{enumerate}
     \item $
\sum \delta A = 0,$ where $A$ is a virtual work
\item 
$\sum W = 0;\ W = \vec{F} \cdot \vec{v} = \vec{F}\vec{v} cos(\vec{F}\vec{v})$, where $W$ is power, $v$ -- virtual velocity
 \end{enumerate}
& To find \textit{one} force or reaction force. & \begin{itemize}
    \item Virtual work has infinitesimal displacements.
    \item System must be in static each time.
\end{itemize}\\
%</sndpvd> 
\rowcolor{LightGray}
Lagrange-d'Alambert principle (General Equation of Dynamics)&
%<*sndgeneqndyn> 
 System & 1 in cartesian, \par $n$ in generalized coordinates & \begin{itemize}
    \item $
\sum \delta A + \sum \delta A^{\Phi} = 0$, \par
 where $A$ is a virtual work
    \item
$\sum W + \sum W^{\Phi} = 0$, \par where $W$ is a power
\end{itemize}  & To find accelerations, motion equations & \  \\
%</sndgeneqndyn> 
Newton-Euler equations &
%<*sndnewtoneuler> 
 System & $6k + \sum_{0}^{k} (6-m_i)$, $k$ -- amount of bodies, $m_i$ -- particular joint d.o.f &
\begin{eqnarray*}\left\{\begin{matrix}
f_i = F_i + f_{i+1} \\
m_i = M_i + m_{i+1} + \vec{p}_{c_i} \times F_i + \vec{p}_{i+1} \times f_{i+1} \\
\tau_i = \left\{\begin{matrix} m_i \cdot Z_i \text{, if revolute} \\ f_i \cdot Z_i \text{, if prismatic} \end{matrix}\right.
\end{matrix}\right.\\
\text{Where} \\
F_i = m \vec{\dot{v}}_{C_i}\\
M_i = I_{C_i}\dot{\omega}_i + \omega_i \times I_{C_i}\omega_i
\end{eqnarray*}
& To find accelerations, forces, motion equations. & We divide a system into separate bodies, write equations for rotations and translations, plus constraints. At the end we have a dozens of equations, which should be solved numerically.\\
%</sndnewtoneuler> 
\rowcolor{LightGray}
Euler-Lagrange equations &
%<*sndeulerlagrange> 
 Body, System & $s$, $s$ -- d.o.f of system &
\begin{eqnarray*}
\frac{\mathrm{d}}{\mathrm{d} t}\left(\frac{\partial T}{\partial \dot{q}_i}\right) - \frac{\partial T}{\partial q_i} = Q_i \text{ or},\\
\frac{\mathrm{d}}{\mathrm{d} t}\left(\frac{\partial L}{\partial \dot{q}_i}\right)  - \frac{\partial L}{\partial q_i}= Q^D_i\\
L = T - \Pi 
\end{eqnarray*}
& To find accelerations, forces, motion equations. & \begin{itemize}
    \item It is the most classical and flexible method for multi-body systems, concluding only holonomic constraints.
    \item $q$ -- generalized coordinates.
    \item $Q$ -- all generalized forces.
    \item $Q^D$ -- non-potential general forces.
\end{itemize}\\
%</sndeulerlagrange> 

\end{longtable}


% \end{landscape}
\end{document}